\documentclass[10pt]{article}
\usepackage[utf8]{inputenc}
\usepackage[left=2cm,right=2cm,top=2cm,bottom=2cm,a4paper]{geometry}
\setlength{\parindent}{0pt}
\setlength{\parskip}{0pt plus 0.5ex}
\usepackage[x11names]{xcolor}
\linespread{1.15} % Line spacing.


% Title sec:
% ==========

\usepackage{titlesec} % !Incompatible with beamer.
\titlelabel{\thetitle.\quad} % Dot after counter
\titleformat*{\section}{\Large\bfseries\color{Firebrick4}}
\titleformat*{\subsection}{\large\bfseries\color{Blue4}}
\titleformat*{\subsubsection}{\normalsize\bfseries}
\titlespacing*{\section}{0mm}{1ex plus 0.2ex minus 0.2ex}{0.5ex plus .1ex}
\titlespacing*{\subsection}{0mm}{1ex plus 0.2ex minus 0.2ex}{0.5ex plus .1ex}
\titlespacing*{\subsubsection}{0mm}{1ex plus 0.2ex minus 0.2ex}{0.5ex plus .1ex}
\setcounter{secnumdepth}{2}


% Mathematics:
% ============

\usepackage{amssymb,amsmath,amsthm,amsfonts}
\usepackage{centernot}
\usepackage{mathtools}
\usepackage{mathrsfs}

\everymath{\displaystyle} % Keeps same font size for inline math. For eg. in fractions.


\newtheoremstyle{mythmstl}{}{}{\itshape}{}{\bfseries}{:}{\newline}{\thmname{#1}\thmnumber{ #2}\thmnote{ (#3)}}%
\theoremstyle{mythmstl}
\newtheorem{theorem}{Theorem}[section]
\theoremstyle{plain} % default
\newtheorem{corollary}{Corollary}[theorem]
\newtheorem{lemma}[theorem]{Lemma}
\newtheorem{proposition}{Proposition}

\newtheoremstyle{mydefstl}{}{}{}{}{\bfseries}{:}{\newline}{\thmname{#1}\thmnumber{ #2}\thmnote{ (#3)}}%
\theoremstyle{mydefstl}
\newtheorem{definition}{Definition}[section]
\newtheorem{example}{Example}[section]
\newtheorem{exercise}{Exercise}[section]

\theoremstyle{remark}
\newtheorem*{remark}{Remark}
\newtheorem*{note}{Note}
\newtheorem*{case}{Case}


% New declarations
% ================
\usepackage{mathtools}
\DeclarePairedDelimiter\ceil{\lceil}{\rceil}
\DeclarePairedDelimiter\floor{\lfloor}{\rfloor}
\DeclarePairedDelimiter\angbrac{\langle}{\rangle}


% Tables and Box:
%================
\usepackage{tabularx} % To use table that wrap text and fit the column width.
\usepackage{multicol,multirow}
\usepackage{booktabs} % Necessary for \toprule \midrule in tables.
\usepackage{tabu} % Better tables.
\usepackage[framemethod=TikZ]{mdframed} % To make text boxes.


% URL
%========
\usepackage{url} % For \url{} Alternate for above.
\usepackage[colorlinks]{hyperref}
% USAGE: \href{url}{alias}
\hypersetup{citecolor=pink}
\hypersetup{linkcolor=red}
\hypersetup{urlcolor=blue}

\usepackage{bbm}


\title{Assignment - 02\\ CS5691 Pattern Recognition and Machine Learning}
\author{Ramasamy Kandasamy\\ CS22M068}
\date{\today}

\begin{document}
\maketitle



\section*{Question - 1}
\textcolor{red}{Code: \texttt{q1.py}}\\
\textcolor{blue}{i. Determine which probabilisitic mixture could have generated this data (It is
not a Gaussian mixture). Derive the EM algorithm for your choice of mixture
and show your calculations. Write a piece of code to implement the algorithm
you derived by setting the number of mixtures K = 4. Plot the log-likelihood
(averaged over 100 random initializations) as a function of iterations.}

\textcolor{red}{Code: \texttt{em\_bernoulli.py}}


Notations:\\
\begin{tabularx}{\linewidth}{lX}
\hline
$N$ & Size of input dataset.\\
$D$ & Dimension of the dataset.\\
$K$ & Number of clusters in the mixture model.\\
$i$ & Index on the datapoints.\\
$j$ & Index for dimension.\\
$k$ & Index for clusters.\\
\hline
\end{tabularx}

My choice of mixture model for this problem is a mixture of four multivariate Bernoulli distribution. This is depicted in the figure below.\\ 


\begin{center}
\includegraphics[width = 0.5\textwidth]{../images/model.jpg}	
\end{center}

In the first step one of the model is chosen with a probability $\pi_k$, and $\sum_{k = 0}^{3} \pi_k = 1$. After one of the four models is chosen we proceed to the second step where the data is generated by the model. Each model has a parameter $p_i$ which is a vector/array of dimension $50$. The input data is a set of array $x_i$ of dimension $50$. For simplicit, I make an assumption that each position within the array is generated independently of each other by a Bernoulli trial with success probability $p_k$. Therefore the probability that a data point $x_i$ is generated by the model $k$ would be:

$$
	P(x_i \mid p_k) = \prod_{j = 1}^D P(x_{i, j} \mid p_{k, j}) 
	= \prod_{j = 1}^D p_{k, j}^{x_{i, j}} \times (1 - p_{k, j})^{1 - x_{i, j}}
$$

The probability of observing the datapoint $x_i$ is given by:

$$
	P(x_i \mid p_0 \cdots p_K) = \sum_{k = 1}^{K} \pi_k \times P(x_i \mid p_k)
	= \sum_{i = 1}^{K} \pi_k \times \prod_{j = 1}^D \left( p_{k, j}^{x_{i, j}} \times (1 - p_{k, j})^{1 - x_{i, j}} \right)
$$

Further, I also assume that each datapoint is generated independently. With this assumption the likelihood of observing the data given paraments is given by:

$$
	L(X\mid \theta) = \prod_{i = 1}^{N}\left(\sum_{k = 1}^{K} \pi_k \times P(x_i \mid p_k)\right)
$$

Log likelihood is given by:

$$
	logL(X \mid \theta) = \log\left(\prod_{i = 1}^{N}\left(\sum_{k = 1}^{K} \pi_k \times P(x_i \mid p_k)\right)\right)
	= \sum_{i=1}^{N}\left(\log\left(\sum_{k = 1}^{K} \pi_k \times P(x_i \mid p_k)\right)\right)
$$

Now, we use Jensen's inequality and use modified log-likelihood instead of the original log-likelihood. Now our goal to maximize this modified log-likelihood which is given by:

$$
mod\_logL(X \mid \theta) = \sum_{i = 1}^{N} \sum_{k = 1}^{K}\left(
	\lambda_i^k\left(
		\sum_{j = 1}^{D} x_{i, j} \log p_{k, j} + (1 - x_{i, j})\log(1-p_{k, j})
\right)
	+ \log \pi_k - \log\lambda_i^k
\right)
$$

First, differentiating this with respect to $p_{k, j}$ gives:\\

$$
	\sum_{i = 1}^{N} \lambda_i^k \left(x_{i, j}\frac{1}{p_{k, j}} 
	- (1 - x_{i, j})\frac{1}{1 - p_{k, j}}\right) = 0.
$$

$$
	\implies \sum_{i = 1}^{N} \lambda_i^k \left(x_{i, j}(1 - p_{k, j})
	- (1 - x_{i, j})p_{k, j}\right) = 0.
$$

Simple algebraic simplification gives:

$$
	p_{k, j} = \frac{\sum_{i = 1}^N \lambda_i^k x_{i, j}}{\sum_{i = 1}^{N} \lambda_i^k}
$$

% In matrix form this would be:

% $$
% 	P = \frac{\Lambda X^T}{\Lambda \mathbbm{1}_{N\times D}}
% $$

where, $P$ is matrix of $p_{k, j}$s, $X$ is the input data and $\mathbbm{1}_{N\times D}$ is a matrix of ones of dimension $N \times D$.

The value of paramenters $\pi_k$ is obtained from $\lambda_k^i$ which inidcates the probability that the $i$-th datapoint was generated by the $k$-th model. Since $\pi_k$ is the probability that any datapoint is generated by the $k$-th model, it is given by:\\

$$
	\pi_k = \sum_{i = 1}^{N} \lambda_k^i
$$

% which in matrix form is

% $$
% 	\Pi = \Lambda \mathbbm{1}_N
% $$

where $\Pi$ is vector of $\pi_{k}$s and $\mathbbm{1}_N$ is a vector of ones of length $N$.\\
Now, $\lambda_k^i$ is the probability that the $i$-th datapoint is generated by $k$-th model. Therefore this is given by:

$$
	\lambda_k^i = \frac{P(x_i \mid p_k) \pi_k}{\sum_{k = 1}^KP(x_i \mid p_k) \pi_k}
$$

These equations are used in  matrix form in the code for EM algorithm, which is in the file \textcolor{red}{\texttt{em\_bernoulli.py}}

\pagebreak

The plot of log-likelihood (averaged over 100 random initializations) as function of iterations is as follows

\begin{center}
	\includegraphics[width=0.5\textwidth]{../plots/q1_i.pdf}	
\end{center}


\vspace*{24pt}
\textcolor{blue}{
ii. Assume that the same data was infact generated from a mixture of Gaussians
with 4 mixtures. Implement the EM algorithm and plot the log-likelihood (aver-
aged over 100 random initializations of the parameters) as a function of iterations.
How does the plot compare with the plot from part (i)? Provide insights that
you draw from this experiment.
}

\textcolor{red}{Code: \texttt{em\_gauss.py}}

Here I assumed that the data was generated by a mixture of multivariate Gaussian distribution. The implementation is in the file \textcolor{red}{\texttt{em\_gauss.py}}.\\
The plot of log-likelihood (averaged over 10 random initializations) as function of iterations is as follows.\\


\begin{center}
	\includegraphics[width=0.5\textwidth]{../plots/q1_ii.pdf}	
\end{center}

The log-likelihood plot reaches a plateau within 100 itertions. However the log-likelihood is around $-2,200$ which significantly higher than what is obtained with the previous model in Q1-i. I think this might be because of overfitting since there are lot more parameters in this model.

\vspace*{24pt}

\textcolor{blue}{iii. Run the K-means algorithm with K = 4 on the same data. Plot the objective of K-means as a function of iterations.}

\textcolor{red}{Code: \texttt{k\_means.py}}\\
The K-means algorithms was run for $K = 4$ and the objective function was plotted.

\begin{center}
	\includegraphics[width=0.5\textwidth]{../plots/q1_iii.pdf}	
\end{center}

\textcolor{blue}{Among the three different algorithms implemented above, which do you think
you would choose to for this dataset and why?}

I would choose the algorithm in Q1-i, which is based on multivariate Bernoulli distribution. This because first mixture models are better than K-means clustering particularly when the data are not clearly separable into clusters. I will choose Q1-i over Q1-ii because according GMM the data should have values other than $0$ and $1$, however this not the case, therefore I will choose the model in Q1-i.


\section*{Question-2}
\textcolor{red}{Code: \texttt{q2.py}}

\textcolor{blue}{Obtain the least squares solution wML to the regression problem using the analytical solution.}

\textcolor{red}{Code: \texttt{q2.py}}

\begin{verbatim}
Q2-i: First five values of w_ml
[[-0.00784961]
 [-0.01367153]
 [-0.00361656]
 [ 0.00264909]
 [ 0.18855145]]
Q2-i: Last five values of w_ml
[[-0.00943541]
 [ 0.01829054]
 [-0.00116999]
 [-0.00261599]
 [-0.00858616]]	
\end{verbatim}

\textcolor{blue}{Code the gradient descent algorithm with suitable step size to solve the least squares algorithms and plot $\mid w^t - w_{ML}\mid ^ 2$ as a function of $t$. What do you observe?}

\textcolor{red}{Code: \texttt{grad\_descent.py}}

The plot of $\mid w^t - w_{ML}\mid ^ 2$ is as follows:

\begin{center}
	\includegraphics[width=0.5\textwidth]{../plots/q2_ii_diff_wt_wml.png}	
\end{center}

$w^t$ approached $w_{ML}$ smoothly, i.e. monotonously.

\vspace*{24pt}

\textcolor{blue}{Code the stochastic gradient descent algorithm using batch size of 100 and plot $\mid w^t - w_{ML}\mid ^ 2$ as a function of t. What are your observations?}

\textcolor{red}{Code: \texttt{stochastic\_grad\_descent.py}}


The plot of $\mid w^t - w_{ML}\mid ^ 2$ is as follows:

\begin{center}
	\includegraphics[width=0.5\textwidth]{../plots/q2_iii_1000_diff_wt_wml.png}	
\end{center}

\begin{center}
	\includegraphics[width=0.5\textwidth]{../plots/q2_iii_10000_diff_wt_wml.png}	
\end{center}

Unlike the normal gradient descent stochastic gradient descent approaches $w_{ML}$ in a not so smooth manner, i.e. the plot is not monotonously decreasing.

\pagebreak

\textcolor{blue}{Code the gradient descent algorithm for ridge regression. Cross-validate for various choices of $\lambda$ and plot the error in the validation set as a function of $\lambda$. For the best $\lambda$ chosen, obtain $w_R$. Compare the test error (for the test data in the file A2Q2Data test.csv) of $w_R$ with $w_{ML}$. Which is better and why?}

\textcolor{red}{Code: \texttt{grad\_descent\_ridge.py}, \texttt{cross\_validation.py}}

The plot of error on validation set for various values of $\lambda$ are as follows


\begin{center}
	\includegraphics[width=0.5\textwidth]{../plots/q2_iv_validation_-20_0_log.png}	
\end{center}


\begin{center}
	\includegraphics[width=0.5\textwidth]{../plots/q2_iv_validation_0_1_lin.png}	
\end{center}

\begin{center}
	\includegraphics[width=0.5\textwidth]{../plots/q2_iv_validation_0_5_lin.png}	
\end{center}

\begin{center}
	\includegraphics[width=0.5\textwidth]{../plots/q2_iv_validation_0_100_lin.png}	
\end{center}

The value of lambda for which the error was minimum was $\lambda = 0$. The test error of $w_R$ and $w_{ML}$ are as follows:

\begin{verbatim}
min lambda:
0.0
Error in test data for w_R is:
184.33438512662875
Error in test data for w_ML is:
185.36365558489564	
\end{verbatim}

Both are equivalent because they have very simillar test error and also in $w_R$ uses $\lambda = 0$ which equivalent to having $w_{ML}$.

\end{document}